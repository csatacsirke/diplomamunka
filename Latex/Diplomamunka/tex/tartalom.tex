
\DeclarePairedDelimiter{\norm}{\lVert}{\rVert}


\newcommand{\sumn}[1]{\sum\limits_{{#1}=1}^{n}}

% Ajánlott minden fő fejezetet külön fájlba írni, pl.:



%\include{tex/bevezeto}
%\include{tex/felhasznaloi}
%\include{tex/fejlesztoi}
%\include{tex/irodalom}



%A diplomamunkának a következő fő részekből kell állnia: 
%1. A dolgozatban megoldott probléma megfogalmazása.
%2. A problémakör irodalmának, az előzményeknek rövid áttekintése.
%3. A probléma megoldásának részletes ismertetése, a választott megoldás indoklása.
%4. Az eredmények összefoglaló értékelése és a levonható következtetések leírása.
%5. 5. Ha a diplomamunka fő eredménye egy program, akkor a dolgozat része a program felhasználói
%dokumentációja, fejlesztési dokumentációja

% III. A diplomamunkára vonatkozó formai követelmények:
% 1. A diplomamunkát nyomtatva, bekötve kell benyújtani az illetékes tanszékre.
% 2. A diplomamunka első oldalán fel kell tüntetni a diplomamunka címét, szerzőjének nevét,
% szakját, az illetékes tanszéket, a témavezető nevét, a külső konzulens nevét, a beadás helyét és
% a védés évét.
% 3. A dolgozat 2. oldala a hivatalos diplomamunkai témabejelentő.
% 4. A Bevezetés tartalmazzon a Diplomamunka-téma bejelentő lapon kitűzött feladat teljesítésének
% mértékére vonatkozó információt.
% 5. A diplomamunka fő részei a dolgozat önálló fejezetei legyenek.
% 6. A diplomamunkának legyen Tartalomjegyzéke és a felhasznált irodalomról Irodalomjegyzéke.
% 7. A diplomamunkában be kell tartani a hivatkozások és idézések standard szabályait. 


We did a pretty good\textsuperscript{\textit{job}}so far

\section{Bevezetés}
\section{Háttér}




nyomdász
vonalvastagság
legkisebb nyomtatható elem
különböző nyomda és fénymásoló technika
akár azt is meg lehet mondani mivel készült
aktív track and trace + biztonsági elemek
-> másfeles kategóriás cuccot szeretnénk
OVD


roc


\subsection{Mikor biztonságos egy nyomat?}

Alapvetően akkor, hogy ha megpróbáljuk valahogy reprodukálni, akkor
mérhető lesz a különbség az eredeti és a hamis között.

Ha védeni szeretnénk egy dokumentumot vagy bankjegyet általában 
többféle módszert használunk egyszerre, amelyek optimális esetben 
ortogonálisak, tehát hamisításnál mindegyikre valahogy máshogyan kell 
felkészülni. Ezek a \textit{feature-öket} alapvetően két kategóriába tudjuk sorolni. 
Az első amit az utca embere is könnyen ellenőrizhet, például a Magyar Forintokon 
a fémszalag, vagy a színváltó tinta, de akár a papír taktilitása is.
A második amihez már valamilyen speciális eszköz is kell. Erre a legismertebb 
példa az UV lámpa, ami mindenhol megtalálható ahol pénz forog.
A valóságban létezik egy harmadik kategória is, ami a titkos, csak 
laborban kimutatható tulajdonságokat tartalmazza, de ezekkel érthető
módon nem foglalkozunk.


\subsection{Előző megoldások}

Ha egy nyomdász vagy szakértő ránéz egy biztonsági nyomatra, 
általában már a hagyományos, tintával készült részekből meg tudja
állapítani az eredetiséget. Ennek oka, hogy a hamisításhoz jó esetben nem áll 
rendelkezésre sem ugyanolyan papír, sem ugyanolyan nyomtató vagy olyan fénymásoló,
amely pont olyan eredményt produkál mint az eredeti.
Ennek feltétele, hogy mikor tervezzük a nyomatot tudjuk azt, hogy
milyen géppel fog készülni, és úgy tervezzük meg a nyomtatandó struktúrákat,
hogy azok éppen csak nyomtathatóak legyenek.


Kézenfekvő, hogy ezt a tudást szerették volna algoritmizálni. Így születtek olyan
programok amik ilyen képek alapján próbálják megállapítani az eredetiséget. 
Ilyenkor készítettek különböző hamisítványokat, és összehasonlították őket az eredetiekkel,
és addig csiszolták az algoritmusokat, amíg azok elfogadható eredményeket nem adtak.


Így születtek olyan megoldások, amik az első és a második fent említett kategória 
közé esnek, azaz az utca emberének lettek tervezve, de mégis céleszközt használva
ellenőrzik a nyomatot, ám ezeket egy okostelefon birtokában bárki elérheti.
Az ötlet, hogy lefényképezzük a dokumentumot, és ezt helyben, egy \textit{appal} elemezzük.


Speciálisan, az általunk részben felhasznált megoldás különböző szempontok alapján méréseket
végez a képen, és néhány (nagyságrendileg 10-12) mérőszámot ad eredményül, melyeket aztán
aggregál, és egy döntést hoz: \textit{Eredeti, Hamis, Nem tudjuk eldönteni biztonsággal}.

A \textit{Nem tudjuk} esetet több dolog is kiválthatja, például nagyon jó hamisítvány is,
de akár egy rossz kép is, például ha remegett az ember keze fényképezéskor.

\subsection{Gépi tanulás}

TODO valamit általánosan a gép tanulásról?

\subsubsection{Támasztóvektor Gép}

A Támasztóvektor Gép (Support Vector Machine, SVM) egy klaszterező eljárás, amivel egy adathalmaz két 
csoportra bontható. Alapja, hogy teret egy hipersíkkal ketté osztjuk úgy, hogy a két csoport előjeles 
távolsága ($ d_1, d_2 $) a hipersíktól maximális legyen. 
Egy csoport és sík távolsága alatt a sík és az ahhoz legközelebbi, az adott csoportba tartozó elem távolságát értjük. 


A két távolság összegét hívjuk \textit{margónak}: $ d = d_1 + d_2 $.

A bemenő adatok vektorait $ \underline{x_i} $-vel, a hozzájuk tartozó kimenetet
$ y_i $-vel, ahol $ y_i=-1 $, ha az első csoportba tartozik, és $ y_i=1 $, ha a másodikba.

A hipersíkot a normálvektorával(súlyával, weight): $ \underline{w} $, és \textit{bias}-szal 
(részlehajlás): $ b $ ábrázoljuk.


Ekkor a sík pontjai $ \underline{x}: \underline{x}^T \cdot \underline{w} - b = 0 $

A csoportok margójához tartozó hipersíkok pedig:

$ \underline{x}: \underline{x}^T \cdot \underline{w} - b = +1 $

$ \underline{x}: \underline{x}^T \cdot \underline{w} - b = -1 $
\\
Ekkor a feladat $ \underline{w} $ és $ b $ meghatározása úgy, hogy 

$ \underline{x}^T \cdot \underline{w} - b \geq +1 $ ha $  y=1 $

$ \underline{x}^T \cdot \underline{w} - b \leq -1 $ ha $  y=-1 $

Ekkor $ d_1 = d_2 = \frac{2}{\norm{\underline{w}}} $  
\\
Mivel $ d $-t szeretnénk maximalizálni, a feladat felírható:

$ y_i \cdot (\underline{x}^T \cdot \underline{w} + b) \geq 1 $

$ \min\limits_{w, b} \norm{\underline{w}} $

Ez azonban csak akkor működik, ha az adat lineárisan szeparálható. Ellenkező esetben
vezessünk be egy olyan költségfüggvényt, ami azt bünteti ami rossz oldalon van.

$ C(\underline{w},b)  = \frac{1}{n} \sum\limits_{i=1}^{n} 
max(0, 1 - y_i(\underline{w} \cdot \underline{x} - b) + \lambda \norm{\underline{w}} $



\noindent
Megmutatható, hogy a feladat megegyezik a következővel (duális probléma): 

$ \max\limits_{c_1 \dots c_n} \sum\limits_{i=1}^{n}c_i -  $
$ \frac{1}{2}\sumn{i}\sumn{j} y_i c_i (x_i \cdot x_j) y_i c_j $

\noindent
ahol $ \forall i: $

$  \sumn{i} c_i y_i = 0 $

$ 0 \leq c_i \leq \frac{1}{2n\lambda} $

\noindent
Ennek megoldására létezik hatékony numerikus módszer.

\noindent
Ekkor a súlyok:

$ \underline{w} = \sumn{i} c_i y_i x_i $

$ b = \underline{w}^T \cdot \underline{x}  - y_i$

\noindent
TODO mennyire részletezzem, hogy ezt hogy kell megoldani?




\paragraph{Nem lineáris feladatok} 


Sok esetben a feladat nem lineárisan szeparálható, viszont ha a problémateret
transzformáljuk, lehet hogy már igen. 

Vezessük be a bázisfüggvényeket, amelyek az eredeti adatot egy másik térbe,
a jellemző(\textit{feature}) térbe képezik:


$ \varphi : \mathbb{R}^n \rightarrow \mathbb{R}^m $

(általában $ m >> n $ )


\noindent
Ekkor a feladat a következőképpen módosul:

$ \max\limits_{c_1 \dots c_n} \sum\limits_{i=1}^{n}c_i -  $
$ \frac{1}{2}\sumn{i}\sumn{j} y_i c_i (\varphi(x_i) \cdot \varphi(x_j)) y_i c_j $

\noindent
és

$ \underline{w} = \sumn{i} c_i y_i \varphi(x_i) $

$ b = \underline{w}^T \cdot \varphi(\underline{x})  - y_i$




\paragraph{Kernel trükk} 

A jellemzőtér nagy, vagy akár végtelen dimenziós is lehet. Bizonyos esetekben a fenti képletben lévő
$ \varphi(x_i) \cdot \varphi(x_j) $ skaláris szorzatot ki tudjuk számolni anélkül, hogy a transzformációt 
elvégeznénk.



\noindent
Vezessük be a \textit{kernel függvényeket}:

$ K(x_i, x_j) = \varphi(x_i)^T \cdot \varphi(x_j) $




















\section{Megoldási módszerek}
\subsection{Támasztóvektor gép}
\subsection{Konvolúciós háló}
\subsection{Autoencoder}
\section{Eredmények}



