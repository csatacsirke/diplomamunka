\documentclass[11pt]{beamer}
\usepackage[utf8]{inputenc}
\usepackage[T1]{fontenc}
\usepackage[magyar]{babel}
%\usetheme{default}
\usetheme{Madrid}

\usepackage[export]{adjustbox}
\usepackage{graphicx}
\graphicspath{{./img/}}


\newcommand{\sumn}[1]{\sum\limits_{{#1}=1}^{n}}


\author{Bálint Márton}
\title[Diplomamunka]{Biztonsági nyomatok eredetiségének ellenőrzése mesterséges neurális hálózatokkal}
%\subtitle{}
%\logo{\includegraphics[width=10pt]{elte-cimer.jpg}}
\institute[ELTE]{Eötvös Loránd Tudományegyetem}
\date{2018. január 29.}
%\subject{}
%\setbeamercovered{transparent}
%\setbeamertemplate{navigation symbols}{}


\begin{document}
	
\begin{frame}[plain]
	\maketitle
\end{frame}

\begin{frame}
	\frametitle{Motiváció}
	
	\begin{itemize}
	\item 
		Hamisítás
		
		A pénzeket, dokumentumokat hamisítják, ezek eredetiségét ellenőrizni kell.
		
	\item 
		Módszerek a hamisítás ellen:
	
		\begin{enumerate}
			\item
			Az utca embere által ellenőrizhető védelem
			
			(Vízjel, hologramcsík, dombornyomás, stb.)
			
			\item
			Speciális eszközzel, képzett emberek által ellenőrizhető védelem
			
			(Pl. UV tinta)
		\end{enumerate}
		
	\item 
		Köztes megoldás:
		
		Az ellenőrizendő nyomatot egy okostelefonnal lefényképezzük, majd azon egy applikáció kiértékeli az eredményt. 
		
			\begin{enumerate}
			\item
			Mindenki számára elérhető
			
			\item
			Szakértelmet visz az ellenőrzésbe
			
		\end{enumerate}
		
		Ilyen alkalmazást fejlesztett például a Jura Trade Kft.\footnote{http://jura.hu}. 
		
		
	\end{itemize}

\end{frame}


\begin{frame}
	\frametitle{Ötlet}
	
	
	\begin{itemize}
		\item 
		Az említett alkalmazás méréseket végez a képen, ezekből 10-20 mérőszámot állít elő, majd ezek alapján ad eredményt, hogy a kép eredeti, hamis, vagy nem tudja eldönteni.
		
		\item 
		A dolgozat célja az volt, hogy ezt az alkalmazást továbbfejlesszük különböző gépi tanulási módszerekkel.
		
		
	\end{itemize}
\end{frame}



\begin{frame}
	\frametitle{Tanító adatok}

	Bementi képek: \textit{(Kb. 1000 darab)}
	\begin{columns}[t]

		\begin{column}{.5\textwidth}
			\includegraphics[width=0.8\textwidth, center]{img/eredeti-pelda.png}
			\centering
			Eredeti
		\end{column}
		\begin{column}{.5\textwidth}
			\includegraphics[width=0.8\textwidth, center]{img/copy-pelda.png}
			
			\centering
			Fénymásolat

		\end{column}				
	\end{columns}

%	
%	\begin{figure}[h]
%		
%		
%		
%		\begin{minipage}[c]{0.5\linewidth}
%			\centering
%			\includegraphics[width=\textwidth]{img/eredeti-pelda.png}
%			\caption{Egy eredeti nyomat.}
%			\label{fig:eredeti.pelda}
%			
%		\end{minipage}\hfill
%		\begin{minipage}[c]{0.5\linewidth}
%			\centering
%			\includegraphics[width=\textwidth]{img/copy-pelda.png}
%			\caption{Egy színes fénymásolat.}
%			\label{fig:copy.pelda}
%			
%		\end{minipage}
%		
%	\end{figure}

\end{frame}

\begin{frame}
	\frametitle{Gépi tanulás}
	\framesubtitle{Használt módszerek}
	
	\begin{itemize}
	\item 
		Felügyelt tanulás
		
		\begin{enumerate}
		\item 
			SVM	
		\item 
			Mély konvolúciós háló
		
		\end{enumerate}
		
	\item 
		Felügyelet nélküli tanulás
		
		\begin{enumerate}
		\setcounter{enumi}{2}
		\item 
			Konvolúciós autoencoder
		\end{enumerate}		
		
	\end{itemize}
\end{frame}




\begin{frame}
	\frametitle{Support Vector Machine (SVM) - Támasztóvektor Gép}
	\framesubtitle{Rövid áttekintés}
		
	Feladat: 
	
	Adott $ \underline{x}_i $ jellemzők, $ \underline{y}_i $ bináris osztályok, 
	határozzunk meg egy $ (\underline{x}, b) $ hipersíkot úgy, hogy: 
	
	\[
	\begin{cases}
	\underline{x}_i^T \underline{w} - b \geq +1, & \text{ ha }  y_i=1, \\
	\underline{x}_i^T \underline{w} - b \leq -1, & \text{ ha }  y_i=-1.
	\end{cases}
	\]	
	
	Ennek megoldásához a következő kifejezést kell minimalizálnunk:
	
	\[
%	L(\underline{\theta}, \underline{x}_1, \dots, \underline{x}_n, y_1, \dots, y_n)  = 
%	\frac{1}{n} 
	\sum\limits_{i=1}^{n} 
	%\max\big(0, 1 - y_i(\underline{w} \cdot \underline{x}_i - b)\big) + \lambda \norm{\underline{w}}_2^2.
	\max\big(0, 1 - y_i(\underline{w}^T \underline{x}_i - b)\big) + \lambda \lvert\underline{w}\rvert_2^2.
	\]
	
	
	
\end{frame}

\begin{frame}
	\frametitle{Support Vector Machine (SVM) - Támasztóvektor Gép}
	\framesubtitle{Rövid áttekintés}
	
	Belátható, hogy ez a következő kvadratikus programozási feladatra vezethető vissza:
	\begin{multline*}
	\max\limits_{\underline{c}} \sum\limits_{i=1}^{n}c_i -  
	\frac{1}{2}\sumn{i}\sumn{j} y_i c_i (\underline{x}_i^T \underline{x}_j) y_i c_j \\
	\text{ s.t. } \quad 
	\sumn{i} c_i y_i = 0, \quad
	0 \leq c_i \leq \frac{1}{2n\lambda}, \quad 
	i=1,\dots,n.
	\end{multline*}
	
	Ekkor a keresett paraméterek:
	\[
	\underline{w} = \sumn{i} c_i y_i \underline{x}_i.
	\]
	
%	Keressünk egy margón lévő $ (\underline{x}_i, y_i) $ párt:
%	\[
%	b = \underline{w}^T \underline{x}_i  - y_i.
%	\]
	
	\textit{Megjegyzés: Ha nem tökéletesen szeparálható a két adathalmaz, akkor $ b $ megfelelő választásával lehet részrehajlást (bias) állítani.}
\end{frame}




\begin{frame}
	\frametitle{SVM - Kernel trükk}
	\framesubtitle{Rövid áttekintés}
	\textbf{Nemlineáris osztályozás:}
	
	Megnöveljük a bemenetünk dimenzióját úgy, hogy új értékeket számolunk ki az előzőek alapján. Ehhez definiáljuk a következő függvényt:
	\[
	\varphi\colon \mathbb{R}^p \rightarrow \mathbb{R}^q, \quad q \gg p.
	\]
	
	Ezután $ x_i $ minta helyett $ \varphi(x_i) $ bemenettel számolunk.
	
	\textbf{Kernel trükk: }
	
	A feladatban szereplő $ k(\underline{x}_i, \underline{x}_j) := \varphi(\underline{x}_i)^T \varphi(\underline{x}_j) $ skaláris szorzatot bizonyos esetekben akkor is ki tudjuk hatékonyan számolni, ha $ x_i $ dimenziója nagy, vagy esetleg végtelen. (Ekkor $ k $-t nevezzük \textit{kernelnek}.)
%	\[
%	k(\underline{x}_i, \underline{x}_j) = \varphi(\underline{x}_i)^T \varphi(\underline{x}_j).
%	\]
	


\end{frame}

%\begin{frame}
%	\frametitle{Support Vector Machine (SVM) - Támasztóvektor Gép}
%	
%	%A cél az volt, hogy az eredeti alkalmazás által számolt jellemzőkből hozzunk döntést.
%	
%
%	\begin{enumerate}
%	\item 
%		
%	\end{enumerate}		
%
%
%\end{frame}


\begin{frame}
	\frametitle{SVM - Alkalmazás}
	
	Mi az SVM-et arra használjuk, hogy az említett alkalmazás által számolt jellemzőkből gépi tanulás segítségével hozzunk döntést.
	

\end{frame}





\begin{frame}

	\centering
	Köszönöm a figyelmet!

\end{frame}





\end{document}